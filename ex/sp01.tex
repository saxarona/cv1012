\documentclass[]{article}

%These tell TeX which packages to use.
\usepackage{array,epsfig}
\usepackage{amsmath}
\usepackage{amsfonts}
\usepackage{amssymb}
\usepackage{amsxtra}
\usepackage{amsthm}
\usepackage{mathrsfs}
\usepackage{color}
\usepackage[spanish, mexico]{babel}
\usepackage[utf8]{inputenc}
\usepackage{enumitem}
\usepackage{helvet}
\usepackage[framed]{matlab-prettifier}
\newcommand{\matlab}[1]{\lstinline[style=Matlab-pyglike]!#1!}

%Here I define some theorem styles and shortcut commands for symbols I use often
\theoremstyle{definition}
\newtheorem{defn}{Definition}
\newtheorem{thm}{Theorem}
\newtheorem{cor}{Corollary}
\newtheorem*{rmk}{Remark}
\newtheorem{lem}{Lemma}
\newtheorem*{joke}{Joke}
\newtheorem{ex}{Example}
\newtheorem*{sol}{Solution}
\newtheorem{prop}{Proposition}

\renewcommand{\familydefault}{\sfdefault}

\newcommand{\lra}{\longrightarrow}
\newcommand{\ra}{\rightarrow}
\newcommand{\surj}{\twoheadrightarrow}
\newcommand{\graph}{\mathrm{graph}}
\newcommand{\bb}[1]{\mathbb{#1}}
\newcommand{\Z}{\bb{Z}}
\newcommand{\Q}{\bb{Q}}
\newcommand{\R}{\bb{R}}
\newcommand{\C}{\bb{C}}
\newcommand{\N}{\bb{N}}
\newcommand{\M}{\mathbf{M}}
\newcommand{\m}{\mathbf{m}}
\newcommand{\MM}{\mathscr{M}}
\newcommand{\HH}{\mathscr{H}}
\newcommand{\Om}{\Omega}
\newcommand{\Ho}{\in\HH(\Om)}
\newcommand{\bd}{\partial}
\newcommand{\del}{\partial}
\newcommand{\bardel}{\overline\partial}
\newcommand{\textdf}[1]{\textbf{\textsf{#1}}\index{#1}}
\newcommand{\img}{\mathrm{img}}
\newcommand{\ip}[2]{\left\langle{#1},{#2}\right\rangle}
\newcommand{\inter}[1]{\mathrm{int}{#1}}
\newcommand{\exter}[1]{\mathrm{ext}{#1}}
\newcommand{\cl}[1]{\mathrm{cl}{#1}}
\newcommand{\ds}{\displaystyle}
\newcommand{\vol}{\mathrm{vol}}
\newcommand{\cnt}{\mathrm{ct}}
\newcommand{\osc}{\mathrm{osc}}
\newcommand{\LL}{\mathbf{L}}
\newcommand{\UU}{\mathbf{U}}
\newcommand{\support}{\mathrm{support}}
\newcommand{\AND}{\;\wedge\;}
\newcommand{\OR}{\;\vee\;}
\newcommand{\Oset}{\varnothing}
\newcommand{\st}{\ni}
\newcommand{\wh}{\widehat}
\newcommand{\markthis}[1]{{\color{blue}\textbf{#1}}}

%Pagination stuff.
\setlength{\topmargin}{-.3 in}
\setlength{\oddsidemargin}{0in}
\setlength{\evensidemargin}{0in}
\setlength{\textheight}{9.in}
\setlength{\textwidth}{6.5in}
\setlength{\itemsep}{0.45in}
\setlength{\parindent}{0pt}
\pagestyle{empty}



\begin{document}

\begin{center}
{\huge Aplicación de Métodos Numéricos al Ambiente Construido (TC1017)}\\[1.5ex]
{\large \textbf{Situación Problema 1: Diseño de la sección transversal de un canal}\\[1.5ex] %You should put your name here
30.04.20} %You should write the date here.
\end{center}

\vspace{0.2 cm}

{%
\small
Este proyecto es \textbf{en equipos}, y consiste en generar \textbf{una memoria de cálculo} sobre la Situación Problema 1 (que puedes revisar en Canvas), además de un reporte \textbf{individual} sobre los resultados y aprendizajes personales.
}

\bigskip

\section{¿Qué debe llevar nuestra Memoria de Cálculo?}

Una memoria de cálculo es un documento donde se muestra a detalle el proceso y los cálculos hechos para obtener la respuesta a algún problema o resolución de algún diseño.

\bigskip

En general, una memoria de cálculo debe contener lo siguiente:

\begin{itemize}
    \item \textbf{Planteamiento del problema}: en donde se detalle la evaluación de la relevancia de las etapas del proceso. Debe reflejar una estructura científica para abordar el problema a resolver. Las características y factores a tomar en cuenta deben determinarse para cada una de las etapas de la situación problema.
    \item \textbf{Objetivo del proyecto}: se formula de manera clara, detallando tanto una acción a realizar, como una evidencia para evaluar el cumplimiento del mismo objetivo. Aquí es buena idea describir cómo se conectan las \textbf{competencias} con el problema, y cómo podrían auto-evaluarlas ustedes mismos.
    \item \textbf{Ejecución de la parte numérica}: se presentan detalladamente cada uno de los cálculos y sus resultados, de manera numérica y con alguna forma de visualización.
    \item \textbf{Conclusión}: un resumen de \textbf{solamente} lo incluido en el documento, sin agregar información adicional.
    \item \textbf{Bibliografía}: Una lista de fuentes confiables y recursos consultados para la generación del documento, usando un formato estándar (APA, MLA, IEEE, ACM, etc.)
\end{itemize}

Recuerden que éste no es un trabajo de investigación, sino el detalle de todo lo realizado para resolver un problema de análisis numérico.
No hay extensión mínima ni extensión límite; sólo incluyan lo necesario para poder cumplir con lo que se pide.

\pagebreak

\section{¿Qué debo incluir en mi evidencia individual?}

La evidencia individual traerá algunos elementos del reporte por equipo, pero tiene algunos requisitos específicos:

\begin{itemize}
    \item \textbf{Análisis del problema} (10 \%): un párrafo breve donde detalles cuál fue el problema y cómo se resolvió.
    \item \textbf{Investigación sobre la ecuación de Manning} (30 \%): MUY BREVE. Una cuartilla \textbf{a lo mucho}, donde expliques \textbf{con tus propias palabras} a qué hace referencia la ecuación y cada uno de sus componentes. No olvides citar adecuadamente tus fuentes.
    \item \textbf{Práctica}: contesta brevemente \textbf{por qué escogieron el método que están usando} (10 \%), \textbf{explica el algoritmo del método}---o sea, cuáles son los pasos a seguir--- (20\%) y muestra \textbf{una corrida del programa} y los resultados finales (20 \%).
    \item \textbf{Reflexión y bibliografía} (10\%): presenta un resumen de los retos y aprendizajes obtenidos de la situación problema. Cita adecuadamente tus referencias usando un formato estándar.
\end{itemize}

Sé muy claro y al punto; no eches rollo. Se espera que la evidencia individual sea \textbf{a lo mucho} de 3 cuartillas.

\end{document}