\documentclass[]{book}

%These tell TeX which packages to use.
\usepackage{array,epsfig}
\usepackage{amsmath}
\usepackage{amsfonts}
\usepackage{amssymb}
\usepackage{amsxtra}
\usepackage{amsthm}
\usepackage{mathrsfs}
\usepackage{color}
\usepackage[spanish, mexico]{babel}
\usepackage[utf8]{inputenc}
\usepackage{enumitem}
\usepackage{helvet}
\usepackage[framed]{matlab-prettifier}
\newcommand{\matlab}[1]{\lstinline[style=Matlab-pyglike]!#1!}

%Here I define some theorem styles and shortcut commands for symbols I use often
\theoremstyle{definition}
\newtheorem{defn}{Definition}
\newtheorem{thm}{Theorem}
\newtheorem{cor}{Corollary}
\newtheorem*{rmk}{Remark}
\newtheorem{lem}{Lemma}
\newtheorem*{joke}{Joke}
\newtheorem{ex}{Example}
\newtheorem*{sol}{Solution}
\newtheorem{prop}{Proposition}

\renewcommand{\familydefault}{\sfdefault}

\newcommand{\lra}{\longrightarrow}
\newcommand{\ra}{\rightarrow}
\newcommand{\surj}{\twoheadrightarrow}
\newcommand{\graph}{\mathrm{graph}}
\newcommand{\bb}[1]{\mathbb{#1}}
\newcommand{\Z}{\bb{Z}}
\newcommand{\Q}{\bb{Q}}
\newcommand{\R}{\bb{R}}
\newcommand{\C}{\bb{C}}
\newcommand{\N}{\bb{N}}
\newcommand{\M}{\mathbf{M}}
\newcommand{\m}{\mathbf{m}}
\newcommand{\MM}{\mathscr{M}}
\newcommand{\HH}{\mathscr{H}}
\newcommand{\Om}{\Omega}
\newcommand{\Ho}{\in\HH(\Om)}
\newcommand{\bd}{\partial}
\newcommand{\del}{\partial}
\newcommand{\bardel}{\overline\partial}
\newcommand{\textdf}[1]{\textbf{\textsf{#1}}\index{#1}}
\newcommand{\img}{\mathrm{img}}
\newcommand{\ip}[2]{\left\langle{#1},{#2}\right\rangle}
\newcommand{\inter}[1]{\mathrm{int}{#1}}
\newcommand{\exter}[1]{\mathrm{ext}{#1}}
\newcommand{\cl}[1]{\mathrm{cl}{#1}}
\newcommand{\ds}{\displaystyle}
\newcommand{\vol}{\mathrm{vol}}
\newcommand{\cnt}{\mathrm{ct}}
\newcommand{\osc}{\mathrm{osc}}
\newcommand{\LL}{\mathbf{L}}
\newcommand{\UU}{\mathbf{U}}
\newcommand{\support}{\mathrm{support}}
\newcommand{\AND}{\;\wedge\;}
\newcommand{\OR}{\;\vee\;}
\newcommand{\Oset}{\varnothing}
\newcommand{\st}{\ni}
\newcommand{\wh}{\widehat}

%Pagination stuff.
\setlength{\topmargin}{-.3 in}
\setlength{\oddsidemargin}{0in}
\setlength{\evensidemargin}{0in}
\setlength{\textheight}{9.in}
\setlength{\textwidth}{6.5in}
\setlength{\itemsep}{0.45in}
\pagestyle{empty}



\begin{document}

\begin{center}
{\huge Aplicación de Métodos Numéricos al Ambiente Construido (CV1012)}\\[1.5ex]
{\large \textbf{Tarea 01}\\[1.5ex] %You should put your name here
31.03.20} %You should write the date here.
\end{center}

\vspace{0.2 cm}

\subsection*{Funciones, ciclos, condiciones y operaciones}

Programa en MATLAB/Octave la siguiente función:

\begin{enumerate}[label=\alph*)]
    \itemsep2.5ex
    \item La Venganza del Hola Pueblo (\matlab{revenge}) es una función que imprime los números del 1 al $x$, pero bajo las siguientes condiciones:
    \begin{itemize}
        \item \textbf{Recibe} un solo parámetro, $x$.
        \item \textbf{Imprime} ``Hola'' \textbf{si} el número a imprimir es divisible entre 3
        \item \textbf{Imprime} ``Pueblo'' \textbf{si} el número a imprimir es divisible entre 5
        \item \textbf{Imprime} ``Hola Pueblo'' \textbf{si} el número a imprimir es divisible entre 3 y entre 5
        \item \textbf{Imprime} el número tal cual, \textbf{si} el número no cumple ninguna de las condiciones de arriba.
    \end{itemize}
\end{enumerate}

Ejemplo: \texttt{revenge(18)} $\Rightarrow$
\begin{lstlisting}[style=Matlab-editor]
    > 1
    > 2
    > Hola
    > 4
    > Pueblo
    > Hola
    > 7
    ...
    > Hola
    > 13
    > 14
    > Hola Pueblo
    > 16
    > 17
    > Hola
\end{lstlisting}

\bigskip

Deberás entregar \textbf{un archivo} de MATLAB con extensión \textbf{.m}: \texttt{revenge.m}.
El archivo debe tener la función y su documentación \textbf{(al inicio)} en el siguiente formato:

\bigskip

\begin{lstlisting}[style=Matlab-editor]
    function revenge(x)
    % NAME: Arturo Gonzalez
    % STUDENT ID: A01170065
    % REVENGE(x)
    % REVENGE does this
    % You can use it like this
    % Write here an example!
    ...
\end{lstlisting}

\pagebreak

{\Large Recomendaciones:}
\begin{itemize}
    \item Si lo crees necesario, haz un diagrama de flujo que te ayude a guiarte en el proceso.
    \item Asegúrate de estar corriendo el MATLAB en el mismo lugar donde guardaste tus archivos.
    \item Asegúrate de que tu archivo tiene nombre en minúsculas, sin espacios ni acentos o símbolos. 
    \item No te olvides de \textbf{documentar tu función} e incluir tu nombre y número de matrícula en el formato especificado arriba.
\end{itemize}
\end{document}