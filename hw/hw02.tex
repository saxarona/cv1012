\documentclass[]{book}

%These tell TeX which packages to use.
\usepackage{array,epsfig}
\usepackage{amsmath}
\usepackage{amsfonts}
\usepackage{amssymb}
\usepackage{amsxtra}
\usepackage{amsthm}
\usepackage{mathrsfs}
\usepackage{color}
\usepackage[spanish, mexico]{babel}
\usepackage[utf8]{inputenc}
\usepackage{enumitem}
\usepackage{helvet}
\usepackage[framed]{matlab-prettifier}
\newcommand{\matlab}[1]{\lstinline[style=Matlab-pyglike]!#1!}

%Here I define some theorem styles and shortcut commands for symbols I use often
\theoremstyle{definition}
\newtheorem{defn}{Definition}
\newtheorem{thm}{Theorem}
\newtheorem{cor}{Corollary}
\newtheorem*{rmk}{Remark}
\newtheorem{lem}{Lemma}
\newtheorem*{joke}{Joke}
\newtheorem{ex}{Example}
\newtheorem*{sol}{Solution}
\newtheorem{prop}{Proposition}

\renewcommand{\familydefault}{\sfdefault}

\newcommand{\lra}{\longrightarrow}
\newcommand{\ra}{\rightarrow}
\newcommand{\surj}{\twoheadrightarrow}
\newcommand{\graph}{\mathrm{graph}}
\newcommand{\bb}[1]{\mathbb{#1}}
\newcommand{\Z}{\bb{Z}}
\newcommand{\Q}{\bb{Q}}
\newcommand{\R}{\bb{R}}
\newcommand{\C}{\bb{C}}
\newcommand{\N}{\bb{N}}
\newcommand{\M}{\mathbf{M}}
\newcommand{\m}{\mathbf{m}}
\newcommand{\MM}{\mathscr{M}}
\newcommand{\HH}{\mathscr{H}}
\newcommand{\Om}{\Omega}
\newcommand{\Ho}{\in\HH(\Om)}
\newcommand{\bd}{\partial}
\newcommand{\del}{\partial}
\newcommand{\bardel}{\overline\partial}
\newcommand{\textdf}[1]{\textbf{\textsf{#1}}\index{#1}}
\newcommand{\img}{\mathrm{img}}
\newcommand{\ip}[2]{\left\langle{#1},{#2}\right\rangle}
\newcommand{\inter}[1]{\mathrm{int}{#1}}
\newcommand{\exter}[1]{\mathrm{ext}{#1}}
\newcommand{\cl}[1]{\mathrm{cl}{#1}}
\newcommand{\ds}{\displaystyle}
\newcommand{\vol}{\mathrm{vol}}
\newcommand{\cnt}{\mathrm{ct}}
\newcommand{\osc}{\mathrm{osc}}
\newcommand{\LL}{\mathbf{L}}
\newcommand{\UU}{\mathbf{U}}
\newcommand{\support}{\mathrm{support}}
\newcommand{\AND}{\;\wedge\;}
\newcommand{\OR}{\;\vee\;}
\newcommand{\Oset}{\varnothing}
\newcommand{\st}{\ni}
\newcommand{\wh}{\widehat}

%Pagination stuff.
\setlength{\topmargin}{-.3 in}
\setlength{\oddsidemargin}{0in}
\setlength{\evensidemargin}{0in}
\setlength{\textheight}{9.in}
\setlength{\textwidth}{6.5in}
% \setlength{\itemsep}{0.45in}
\pagestyle{empty}



\begin{document}

\begin{center}
{\huge Aplicación de Métodos Numéricos al Ambiente Construido (CV1012)}\\[1.5ex]
{\large \textbf{Tarea 02}\\[1.5ex] %You should put your name here
15.05.20} %You should write the date here.
\end{center}

\vspace{0.2 cm}

Esta actividad es en \textdf{parejas}. Lee las instrucciones y sube un archivo comprimido que tenga:
\begin{itemize}
    \item Los archivos necesarios para realizar los cálculos usando las fórmulas de clase (MATLAB)
    \item Un PDF \textdf{muy breve} con las respuestas al set de ejercicios y una gráfica de cada una de las funciones.
\end{itemize}

\section*{Regresión lineal y mínimos cuadrados}

\begin{enumerate}%[label=\alph*)]
    \itemsep7ex
    \item Usa regresión lineal por mínimos cuadrados para aproximar una línea recta que determine el comportamiento de la tabla siguiente:
    \begin{table}[htbp]
        \centering
        \begin{tabular}{c|cccccccccc}
        $x$ & 0 & 2 & 4 & 6 & 9 & 11 & 12 & 15 & 17 & 19 \\ \hline
        $y$ & 5 & 6 & 7 & 6 & 9 & 8  & 7  & 10 & 12 & 12
        \end{tabular}
    \end{table}
    Y contesta lo siguiente:
    \begin{enumerate}
        \item ¿Cuántos datos tiene la muestra?
        \item ¿Cuál es la media de la muestra?
        \item ¿Cuál es la ecuación de la recta que aproxima el comportamiento de estos datos?
        \item ¿Cuál es la $R^2$ de esta regresión?
        \item Genera un gráfico en MATLAB donde se aprecien los puntos de tu muestra, una recta marcando la media, y la recta de la regresión
    \end{enumerate}
    \item Usa regresión lineal por mínimos cuadrados para aproximar una línea recta que determine el comportamiento de la tabla siguiente:
    \begin{table}[htbp]
        \centering
        \begin{tabular}{c|ccccccccccc}
        $x$ & 6  & 7  & 11 & 15 & 17 & 21 & 23 & 29 & 29 & 37 & 39 \\ \hline
        $y$ & 29 & 21 & 29 & 14 & 21 & 15 & 7  & 7  & 13 & 0  & 3 
        \end{tabular}
    \end{table}
    Y contesta lo siguiente:
    \begin{enumerate}
        \item ¿Cuál es la media de la muestra?
        \item ¿Cuál es la desviación estándar de la muestra?
        \item ¿Cuál es la ecuación de la recta que aproxima el comportamiento de estos datos?
        \item ¿Cuál es la $R^2$ de esta regresión?
        \item Si alguien hiciera una medición de $x=10, y=10$, ¿Pensarías que es una medición válida o que hubo un error en la medición? Justifica tu respuesta.
        \item ¿Cuál sería la medición para $x=18$ y para $x=30$?
        \item Genera un gráfico en MATLAB donde se aprecien los puntos de tu muestra, una recta marcando la media, y la recta de la regresión
    \end{enumerate}
\end{enumerate}

\pagebreak

{\Large Recomendaciones:}
\begin{itemize}
    \item Si lo creen necesario, hagan un diagrama de flujo que los ayude a guiarse en el proceso
    \item Asegúrense de estar corriendo el MATLAB en el mismo lugar donde guardaron sus archivos
    \item Asegúrense de que su archivo tiene nombre en minúsculas, sin espacios ni acentos o símbolos
    \item No se olviden de \textbf{documentar su función} e incluir sus nombres y matrículas
\end{itemize}
\end{document}