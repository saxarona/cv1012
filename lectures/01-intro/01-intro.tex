\documentclass[spanish, c]{beamer}

\usepackage[utf8]{inputenc}
%\usepackage[spanish, mexico]{babel}
\usepackage{amsmath}
\usepackage{mathtools}
\usepackage{hyperref}
\usepackage{xcolor}
\usepackage{color}
\usepackage{ragged2e}
\usepackage{mathrsfs}
\usepackage{csquotes}
\usepackage{listings}
\usepackage[scaled]{beramono}
\usepackage[T1]{fontenc}
\usepackage{matlab-prettifier}
\usepackage{graphicx}
\usepackage{booktabs}

\renewcommand{\indent}{\hspace*{2em}}

% \usepackage{tikz}

% \usetikzlibrary{fit, shapes, arrows}

% \usepackage{courier}
% \usepackage{subfigure}
% \usepackage{enumerate}
% \usepackage{algorithmic}
% \usepackage{algorithm}

% \usepackage{listings}
% \usepackage{lstlinebgrd}

\usetheme{Boadilla}
\usefonttheme[onlymath]{serif}

\newcommand{\matlab}[1]{\lstinline[style=Matlab-editor]!#1!}
\newcommand\blfootnote[1]{%
\begingroup
\renewcommand\thefootnote{}\footnote{#1}%
\addtocounter{footnote}{-1}%
\endgroup
}

\lstset
{
    language = Matlab,
    style = Matlab-editor,
    basicstyle = \mlttfamily\scriptsize,
    escapechar = `,
    numbers = left,
    frame = tb,
}

\lstdefinestyle{output}
{
    language = {},
    basicstyle = \mlttfamily\scriptsize,
    escapechar = `,
    numbers = none,
    showtabs = false,
   	showstringspaces = false,
}

% Sets the templates
\definecolor{navyblue}{RGB}{0, 0, 128}
\definecolor{crimson}{RGB}{128, 16, 0}

\setbeamertemplate{navigation symbols}{}
\setbeamertemplate{headline}{}
\setbeamertemplate{title page}[default][colsep=-4bp,rounded=true]
\setbeamertemplate{footline}[frame number]
\setbeamertemplate{bibliography item}[text]
\setbeamertemplate{theorems}[numbered]

\setbeamercolor{title}{fg=navyblue, bg=white}
\setbeamercolor{frametitle}{fg=navyblue, bg=white}
\setbeamercolor{structure}{fg=navyblue}
\setbeamercolor{button}{fg=white,bg=navyblue}

\setbeamercovered{transparent}

\title{Datos y Matrices}
\subtitle{Aplicación de Métodos Numéricos al Ambiente Construido \\ (CV1012)}
\author{
    \texorpdfstring{
        \begin{center}
            M.C. Xavier Sánchez Díaz \\
            \href{mailto:sax@tec.mx}{\texttt{sax@tec.mx}}
        \end{center}
    }
    {M.C. Xavier Sánchez Díaz}
}

\institute[Tecnológico de Monterrey]{\includegraphics[scale=0.5]{../img/logo}}
\date{}

\begin{document}

\setlength{\rightskip}{0pt}

\begin{frame}[plain]
    \titlepage        
\end{frame}

\begin{frame}{Outline}
    \tableofcontents
\end{frame}

\section{Introducción a métodos numéricos}

\subsection{Continuo vs. Discreto}

\begin{frame}{Continuo vs. Discreto}{Introducción a métodos numéricos}

    \begin{center}
        \LARGE
        ¿Por qué suenan tan distinto los violines de los pianos si ambos instrumentos son de cuerda?
    \end{center} \pause

    \bigskip

    \begin{center}
        \textit{Story time: continuo vs. discreto}
    \end{center}

\end{frame}

\subsection{¿Qué son los métodos numéricos?}

\begin{frame}{¿Qué son los métodos numéricos?}{Introducción a métodos numéricos}
    \begin{block}{Definición}
        Los métodos numéricos son \alert{técnicas} mediante las cuales es posible \textbf{formular problemas matemáticos} de tal forma que puedan resolverse utilizando \textbf{operaciones aritméticas}.
    \end{block} \pause

    \bigskip

    ¿Qué problemas se te vienen a la mente que puedan resolverse con métodos numéricos?
\end{frame}

\begin{frame}{Métodos numéricos: aplicación}{Introducción a métodos numéricos}
    
    Podemos usarlos para muchos problemas en ingeniería:

    \bigskip
    
    \begin{itemize}[<+->]
        \item \textbf{Encontrar} raíces de una ecuación
        \item \textbf{Resolver} sistemas de ecuaciones
        \item \textbf{Encontrar} el valor óptimo en una función
        \item \textbf{Aproximar} funciones
        \item \textbf{Interpolar} valores intermedios usando referencias
        \item \textbf{Integrar}  y \textbf{derivar} funciones
    \end{itemize}
\end{frame}

\subsection{Representación digital de los números}

\begin{frame}{Representación de los números}{Introducción a métodos numéricos}

    Si descomponemos un número en cifras, podemos obtener el valor posicional de cada una para calcular el valor de la misma\dots \pause

    El número $2357$ es:

    \begin{itemize}
        \item $2000 = 2 \times 10^{\alert<7>{3}}$ más\dots \pause
        \item $300 = 3 \times 10^{\alert<7>{2}}$ más\dots \pause
        \item $50 = 5 \times 10^{\alert<7>{1}}$ más\dots \pause
        \item $7 = 7 \times 10^{\alert<7>{0}}$ \pause
    \end{itemize}

    \bigskip

    ¿Qué pasa con los exponentes en nuestro sistema decimal? \pause

\end{frame}

\begin{frame}{Representación de los números}{Introducción a métodos numéricos}

    \begin{center}
        \LARGE
        ¿Qué pasa con los números decimales entonces?
    \end{center}\pause
    
    \bigskip
    
    \begin{center}
        \textit{\href{https://en.wikipedia.org/wiki/Floating-point\_arithmetic}{Wiki: IEEE floating points}}
    \end{center}
    
\end{frame}

\begin{frame}{Tipos de datos}{Introducción a los métodos numéricos}
    
   Existen distintos \alert{tipos de datos} con los que podemos trabajar en una computadora: \pause

    \bigskip

    \begin{itemize}[<+->]
        \item Números enteros (\textit{integer numbers})
        \item Números decimales (\textit{floating point numbers})
        \item Cadenas de caracteres alfanuméricos (\textit{strings})
    \end{itemize} \pause

    \bigskip
    
    En este curso sólo nos preocuparemos por números. \pause

    Cuando \textbf{operamos} con estos datos, generamos nueva información que podríamos necesitar en el futuro.

\end{frame}

\begin{frame}{Datos como resultados}{Introducción a los métodos numéricos}
    Antes de usar la computadora o la calculadora para hacer cálculos, solíamos hacer las operaciones a mano.

    \bigskip
    
    Por ejemplo, si queremos calcular $1270 \times 35$, una manera de hacerlo podría ser\dots

\end{frame}

\begin{frame}{Datos como resultados}{Introducción a los métodos numéricos}
    \begin{align*}
        1270 \times 35 & = \\ \pause
        & = (1200 + 70) \times (7)(5) \\ \pause
        & = (\alert<4>{12})(\alert<5,6>{7})(\alert<4>{5})(\alert<7>{100}) + (\alert<8>{7})(\alert<8>{7})(\alert<10>{5})(\alert<9>{10}) \pause
    \end{align*}
    \vspace{-2ex}
    \begin{align*}
        \alert<4>{12} \times \alert<4>{5} = \alert<5,6>{60} \\
        \alert<5,6>{60} \times \alert<5,6>{7} = \alert<6>{6} \times \alert<6>{7} \times \alert<6>{10} = \alert<7>{420} \\
        \alert<7>{420} \times \alert<7>{100} = \alert<11>{42000} \\[1.5ex]
        \alert<8>{7} \times \alert<8>{7} = \alert<9>{49} \\
        \alert<9>{49} \times \alert<9>{10} = \alert<10>{490} \\
        \alert<10>{490} \times \alert<10>{5} = \alert<10>{490} \times \alert<10>{10 / 2} = \alert<11>{2450}\\[1.5ex]
        \alert<11>{42000} + \alert<11>{2450} = 44450 \quad \square
    \end{align*}
\end{frame}

\begin{frame}{Datos y errores}{Introducción a los métodos numéricos}
    ¿Qué habría pasado si hubiera muchos puntos decimales de por medio? \pause

    \bigskip

    ¿Cuál es la representación decimal de $\dfrac{1}{7}$? ¿Y de $\dfrac{2}{7}$? \pause

    \bigskip

    \begin{center}
        \LARGE
        \textit{Story time: The Wolf}
    \end{center}    
\end{frame}

\begin{frame}{Datos y errores}{Introducción a los métodos numéricos}
    {\Huge \texttt{NotImplementedError}}
\end{frame}

\begin{frame}{Tipos de Errores}{Introducción a los métodos numéricos}
    {\Huge \texttt{NotImplementedError}}
\end{frame}

% What is control flow
% why is it important
% does it exist in math?
% how to represent it
% how to represent it in matlab
% practical cases

% \section*{Referencias}

% \begin{frame}[t]{Referencias}
    % \nocite{bibID01}
    % \nocite{bibID02}

    % \bibliographystyle{IEEE}
    % \bibliography{biblio}
% \end{frame}

\end{document}