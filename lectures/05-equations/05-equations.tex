\documentclass[spanish, c]{beamer}

\usepackage[utf8]{inputenc}
%\usepackage[spanish, mexico]{babel}
\usepackage{amsmath}
\usepackage{mathtools}
\usepackage{hyperref}
\usepackage{xcolor}
\usepackage{color}
\usepackage{ragged2e}
\usepackage{mathrsfs}
\usepackage{csquotes}
\usepackage{listings}
\usepackage[scaled]{beramono}
\usepackage[T1]{fontenc}
\usepackage{matlab-prettifier}
\usepackage{graphicx}
\usepackage{booktabs}

\renewcommand{\indent}{\hspace*{2em}}

% \usepackage{tikz}

% \usetikzlibrary{fit, shapes, arrows}

% \usepackage{courier}
% \usepackage{subfigure}
% \usepackage{enumerate}
% \usepackage{algorithmic}
% \usepackage{algorithm}

% \usepackage{listings}
% \usepackage{lstlinebgrd}

\usetheme{Boadilla}
\usefonttheme[onlymath]{serif}

\newcommand{\matlab}[1]{\lstinline[style=Matlab-editor]!#1!}
\newcommand\blfootnote[1]{%
\begingroup
\renewcommand\thefootnote{}\footnote{#1}%
\addtocounter{footnote}{-1}%
\endgroup
}

\lstset
{
    language = Matlab,
    style = Matlab-editor,
    basicstyle = \mlttfamily\scriptsize,
    escapechar = `,
    numbers = left,
    frame = tb,
}

\lstdefinestyle{output}
{
    language = {},
    basicstyle = \mlttfamily\scriptsize,
    escapechar = `,
    numbers = none,
    showtabs = false,
   	showstringspaces = false,
}

\makeatletter
\renewcommand*\env@matrix[1][*\c@MaxMatrixCols c]{%
  \hskip -\arraycolsep
  \let\@ifnextchar\new@ifnextchar
  \array{#1}}
\makeatother

% Sets the templates
\definecolor{navyblue}{RGB}{0, 0, 128}
\definecolor{crimson}{RGB}{128, 16, 0}

\setbeamertemplate{navigation symbols}{}
\setbeamertemplate{headline}{}
\setbeamertemplate{title page}[default][colsep=-4bp,rounded=true]
\setbeamertemplate{footline}[frame number]
\setbeamertemplate{bibliography item}[text]
\setbeamertemplate{theorems}[numbered]

\setbeamercolor{title}{fg=navyblue, bg=white}
\setbeamercolor{frametitle}{fg=navyblue, bg=white}
\setbeamercolor{structure}{fg=navyblue}
\setbeamercolor{button}{fg=white,bg=navyblue}

\setbeamercovered{transparent}

\title{Ecuaciones, propiedades y eliminación}
\subtitle{Aplicación de Métodos Numéricos al Ambiente Construido \\ (CV1012)}
\author{
    \texorpdfstring{
        \begin{center}
            M.C. Xavier Sánchez Díaz \\
            \href{mailto:sax@tec.mx}{\texttt{sax@tec.mx}}
        \end{center}
    }
    {M.C. Xavier Sánchez Díaz}
}

\institute[Tecnológico de Monterrey]{\includegraphics[scale=0.5]{../img/logo}}
\date{}

\begin{document}

\setlength{\rightskip}{0pt}

\begin{frame}[plain]
    \titlepage        
\end{frame}

\begin{frame}{Outline}
    \tableofcontents
\end{frame}

\section{¿Ecuaciones y matrices?}

\begin{frame}{Ensamblando Robots}{¿Ecuaciones y matrices?}

    \textit{IntelliCorp} produce dos tipos de procesadores, el \texttt{x230} y el \texttt{x260} para sus robots.
    
    Para poder fabricarlos, se necesitan \textbf{silicio}, \textbf{cobre} y \textbf{aluminio}.

    El \texttt{x230} usa 4, 3 y 5 láminas, respectivamente, mientras que el \texttt{x260} usa 5, 2 y 6 placas.

    \begin{center}
        \begin{table}[H]
            \begin{tabular}{@{}ccc@{}}
            \toprule
                                    & x230 & x260 \\ \midrule
            \multicolumn{1}{c|}{Si} & 4    & 5    \\
            \multicolumn{1}{c|}{Cu} & 3    & 2    \\
            \multicolumn{1}{c|}{Al} & 5    & 6    \\ \bottomrule
            \end{tabular}
        \end{table}
    \end{center}
\end{frame}

\begin{frame}{¿Qué tiene más sentido?}{¿Ecuaciones y matrices?}

    ¿Que cada placa se haga con distintos procesadores?

    \begin{align*}
        S & = 4x_1 + 5 x_2 \\
        C & = 3x_1 + 2 x_2 \\
        A & = 5x_1 + 6 x_2
    \end{align*} \pause

    \bigskip

    ¿Que cada procesador se haga con distintas placas?
    
    \begin{align*}
        x_{230} & = 4s + 3c + 5a \\
        x_{260} & = 5s + 2c + 6s
    \end{align*}

\end{frame}

\begin{frame}{Transpuesta}{¿Ecuaciones y matrices?}
    Para que tenga más sentido, podemos \alert{transponer} la matriz.
    Para eso, reescribiremos las \textbf{columnas} de la matriz \textbf{como renglones} y los \textbf{renglones como columnas}:

    \[A = 
        \begin{bmatrix*}
            4 & 5 \\
            3 & 2 \\
            5 & 6
        \end{bmatrix*}
    \]

    \bigskip

    \[A^T = 
        \begin{bmatrix*}
            4 & 3 & 5 \\
            5 & 2 & 6
        \end{bmatrix*}
    \]

\end{frame}

\begin{frame}{De las operaciones al álgebra}{¿Ecuaciones y matrices?}

    Desde ahora, nuestra $A^T$ será $C= \begin{bmatrix*}
        4 & 3 & 5 \\
        5 & 2 & 6
    \end{bmatrix*}$

    \bigskip

    ¿Cuántas placas necesitaríamos para hacer 3 procesadores de cada tipo? \pause

    \bigskip

    \[
        3C = 3 \begin{bmatrix*}
        4 & 3 & 5 \\
        5 & 2 & 6
    \end{bmatrix*} = \begin{bmatrix*}
    12 & 9 & 15 \\
    15 & 6 & 18
\end{bmatrix*} \blacksquare
    \]    
\end{frame}

\begin{frame}{De las operaciones al álgebra}{¿Ecuaciones y matrices?}

    Si la nueva tecnología antiestática utiliza 1 placa adicional de cada material para el \texttt{x230}, y 2 placas de silicio, 1 de cobre y 1 de aluminio adicionales para el \texttt{x260}, ¿cuántas placas necesitaré de ahora en adelante si ahora todos mis procesadores incluirán tecnología antiestática?

    \bigskip \pause

    Primero, ¿cómo es la matriz de costos de la tecnología antiestática? \pause
    
    $$S = \begin{bmatrix*}
        1 & 1 & 1 \\
        2 & 1 & 1
    \end{bmatrix*}$$

    \bigskip \pause

    El nuevo costo entonces es \bigskip

    \[%
        C + S =%
        \begin{bmatrix*}
            4 & 3 & 5 \\
            5 & 2 & 6
        \end{bmatrix*}
        %
        +
        %
        \begin{bmatrix*}
            1 & 1 & 1 \\
            2 & 1 & 1
        \end{bmatrix*}
        %
        =
        %
        \begin{bmatrix*}
            5 & 4 & 6 \\
            7 & 3 & 7
        \end{bmatrix*} \blacksquare
    \]
\end{frame}

\begin{frame}{De las operaciones al álgebra}{¿Ecuaciones y matrices?}
    Nuestra nueva $C$ es ahora
    $C = \begin{bmatrix*}
        5 & 4 & 6 \\
        7 & 3 & 7
    \end{bmatrix*}$.
    Si sabemos que cada placa de silicio cuesta \$4, cada placa de cobre \$2 y cada placa de aluminio \$3, ¿Cuál es el precio total de cada procesador en \$? \pause

    \bigskip

    Nuestro vector de precios es $\mathbf{p} = [4, 2, 3]^T$ así que\dots \pause

    \bigskip

    \begin{align*}
        C\mathbf{p} =
        \begin{bmatrix*}
            5 & 4 & 6 \\
            7 & 3 & 7
        \end{bmatrix*}
        \begin{bmatrix*}
            4 \\ 2 \\ 3    
        \end{bmatrix*} & = 4 \begin{pmatrix*} 5 \\ 7 \end{pmatrix*} + 2 \begin{pmatrix*} 4 \\ 3 \end{pmatrix*} + 3 \begin{pmatrix*} 6 \\ 7 \end{pmatrix*} \\
        & = \begin{bmatrix*} 20 + 8 + 18 \\ 28 + 6 + 21 \end{bmatrix*} \\
        & = \begin{bmatrix*} 46 \\ 55 \end{bmatrix*} \blacksquare
    \end{align*}

\end{frame}

\begin{frame}{De las operaciones al álgebra}{¿Ecuaciones y matrices?}
    Ya sabemos el precio de cada procesador. Ahora queremos saber la resistencia eléctrica de cada uno, así como también su peso. \pause

    La matriz que contiene esta información (la columna de resistencias y la columna de pesos, por cada material) es
    $D = \begin{bmatrix*}
        3 & 2 \\
        1 & 4 \\
        2 & 3
    \end{bmatrix*}$

    \bigskip
    
    ¿Puedo hacer la multiplicación de siempre? ¿Qué matriz obtendré? \pause

    \[
        CD = \begin{bmatrix*}
            31 & 44 \\
            38 & 47
        \end{bmatrix*}
    \]

    Que es la matriz de resistencia y peso (heredados de $D$) de los procesadores (heredados de $C$).
\end{frame}

\begin{frame}{De las operaciones al álgebra}{¿Ecuaciones y matrices?}
    Volvamos a transponer nuestra matriz para poder manejar pedidos \\
    (materiales~$\times$~procesador y procesadores~$\times$~pedido para obtener\\
    materiales $\times$ pedidos):
    $C = C^T = 
    \begin{bmatrix}
        5 & 7 \\
        4 & 3 \\
        6 & 7
    \end{bmatrix}$
    
    \bigskip

    Y asumamos que nos hacen un pedido de 3 procesadores \texttt{x230} y 3 procesadores \texttt{x260}.
    ¿Cuánto material necesito? \pause

    \begin{itemize}[<+->]
        \item ¿Cuál es el vector que representa un pedido de 2 y 0 \texttt{x230} y \texttt{x260} respectivamente?
        \item ¿Y si nuestro pedido fuera de 2 y 2?
        \item ¿Y si fuera de 2 y 3?
        \item ¿Y si mi pedido fuera de 1 y 1?
    \end{itemize}
\end{frame}

\begin{frame}{Matriz escalar}{¿Ecuaciones y matrices?}
    Una \alert{matriz escalar} es una matriz que sólo tiene \textbf{escalares} en la \textbf{diagonal}. Sirven para \textit{escalar} una matriz: cada una de las columnas por un \textit{cierto} factor. \pause

    \bigskip

    ¿Cuál es el resultado de la siguiente operación? \bigskip

    \[
        \begin{bmatrix}
            \color{red} 5 & \color{blue} 7 \\
            \color{red} 4 & \color{blue} 3 \\
            \color{red} 6 & \color{blue} 7
        \end{bmatrix}
        \begin{bmatrix*}
            \color{red} 2 & 0 \\
            0 & \color{blue} 3
        \end{bmatrix*} = \pause
        \begin{bmatrix*}
            \color{red} 10 & \color{blue} 21 \\
            \color{red} 8 & \color{blue} 9 \\
            \color{red} 12 & \color{blue} 21
        \end{bmatrix*}
    \]
\end{frame}

\begin{frame}{Matriz identidad}{Ecuaciones y matrices}
    De las matrices \textbf{escalares}, aquella que al multiplicarla por cualquier matriz $A$ hace que se cumpla \textit{la identidad multiplicativa}, dando como resultado $A$, se le conoce como \alert{matriz identidad}. \pause

    \bigskip

    \[
        \begin{bmatrix*}
            10 & 21 \\
            8 & 9 \\
            12 & 21
        \end{bmatrix*}
        \begin{bmatrix*}
            1 & 0  \\
            0 & 1
        \end{bmatrix*} = \begin{bmatrix*}
            10 & 21 \\
            8 & 9 \\
            12 & 21
        \end{bmatrix*}
    \] \pause

    \bigskip

    Es decir, una \textbf{matriz escalar} con 1 en la diagonal y 0 en el resto de las posiciones. Para representarla usamos la letra $I$, de $I$\textit{dentity}. \pause

    \bigskip

    ¿Cuáles serán las dimensiones de una \alert{matriz identidad} que queremos multiplicar por una matriz de $4 \times 3$?
\end{frame}

\section{Ecuaciones y matrices}

\begin{frame}{Álgebra de matrices}{Ecuaciones y matrices}

    Asumamos que tenemos $AX = B$, donde tanto $A$ como $B$ y $X$ son matrices. ¿Cómo hacemos para despejar la matriz $X$?

    \bigskip

    \dots {\scriptsize (no, no es dividiendo. Recordemos que $AB$ no es lo mismo que $BA$)}

    \bigskip

    Si la multiplicación de matrices es una \textit{transformación lineal} que aplicamos a otra matriz\dots debe existir algo que la revierta, ¿no? \dots

    \bigskip

    \dots Sí, pero no siempre.

\end{frame}

\begin{frame}{Ecuaciones lineales}{Ecuaciones y matrices}
    Una \alert{ecuación} es \textbf{una igualdad matemática}. En este curso nos enfocaremos en aquellas igualdades que son \textbf{lineales}, es decir que todas sus \textbf{variables} son de \textit{grado 1} (o con exponente 1). \pause

    \bigskip

    \[3x + 2y + 4z = 250\] \pause

    Una ecuación es una manera muy conveniente de expresar, por ejemplo, que 3 $cocas$, 2 $burgers$ y 4 $papas$ salen \$250. \pause

    Si vamos en repetidas ocasiones (y los precios no cambian), podríamos generar más ecuaciones, dependiendo de los pedidos que hagamos, por ejemplo $2x + 2y + 2z = 190$.
\end{frame}

\begin{frame}{Disectando una ecuación}{Ecuaciones y matrices}

    \[\alert<4>{3}\alert<3>{x} + \alert<4>{2}\alert<3>{y} + \alert<4>{4}\alert<3>{z} = \alert<5>{250}\] \pause

    \begin{itemize}[<+->]
        \item \alert<3>{Variables} o \textbf{incógnitas}
        \item \alert<4>{Coeficientes}
        \item \alert<5>{Término constante}
    \end{itemize} \pause

    Si la reescribimos con el término constante del lado izquierdo, podemos obtener:

    \[3x + 2y + 4z -250 = 0\]

    que se conoce como \textbf{ecuación homogénea}.  
\end{frame}

\begin{frame}{Matrices para ecuaciones}{Ecuaciones y matrices}

    Podemos usar matrices para representar un \alert{sistema de ecuaciones lineales}, que son múltiples ecuaciones \textbf{simultáneas}: \pause

    \begin{align*}
        & 3x + 2y + 4z = 270 \\
        & 2x + 2y + 2z = 200 \\
        & 4x + 4y + 3z = 375
    \end{align*} \pause

    Podemos hacer una matriz con los coeficientes, y un vector con los resultados, para luego unirlos en una \alert{matriz aumentada}. Si 

    \[
        C = \begin{bmatrix*}
            3 & 2 & 4 \\
            2 & 2 & 2 \\
            4 & 4 & 3
        \end{bmatrix*} \text{ y }
        \mathbf{r} =
        \begin{bmatrix*}
            270 \\ 200 \\ 375
        \end{bmatrix*} \quad \text{entonces} \quad
        M = \begin{bmatrix}[ccc|c]
            3 & 2 & 4 & 270 \\
            2 & 2 & 2 & 200 \\
            4 & 4 & 3 & 375
        \end{bmatrix}    
    \]   

\end{frame}

\begin{frame}{Resolviendo sistemas de ecuaciones}{Ecuaciones y matrices}
    Vamos a \alert{resolver} \textbf{el sistema} para encontrar cuánto valen $x, y$ y $z$.
    Para eso, sólo podemos usar las siguientes \alert{operaciones por renglón}: \pause

    \bigskip

    \begin{enumerate}[<+->]
        \item Intercambiar renglones de lugar: $$R_1 \leftrightarrow R_2$$
        \item Escalar renglones (multiplicando por escalar): $$R_1 \leftarrow cR_1$$
        \item A un renglón existente, sumarle o restarle otro renglón escalado: $$R_1 \leftarrow R_1 - cR_2$$
    \end{enumerate}
\end{frame}

\begin{frame}{Aplicando las reglas de manera ordenada}{Ecuaciones y matrices}
    El \textbf{algoritmo de} \alert{eliminación Gaussiana} resuelve ecuaciones lineales de la siguiente forma:

    \begin{enumerate}
        \item Prepare la matriz (ordene los renglones para tener el $a_{11}$ distinto de 0)
        \item Obtenga 0s debajo de $a_{11}$ usando las reglas anteriores.
        \item Olvídese de la columna 1 y fila 1. Pase al $a_{22}$ y \textbf{repita} desde el paso 1 \textbf{hasta} que tenga una \alert{matriz escalonada}.
        \item Obtenga 0s arriba de $a_{mn}$ usando las reglas anteriores.
        \item Olvídese de la columna $m$ y la fila $n$. Pase al $a_{m-1n-1}$ y repita desde el paso 4 y hasta que consiga la \textbf{matriz identidad}.
        \item Usted ha terminado.
    \end{enumerate}

\end{frame}

\begin{frame}{Eliminación Gaussiana}{Ecuaciones y Matrices}

    \begin{align*}
        M =
    & \begin{bmatrix}[ccc|c]
        3 & 2 & 4 & 270 \\
        2 & 2 & 2 & 200 \\
        4 & 4 & 3 & 375
    \end{bmatrix}
    & \hfill R_2 \gets \frac{R_2}{2}, R_1 \longleftrightarrow R_2 \\
    = & \begin{bmatrix}[ccc|c]
        1 & 1 & 1 & 100 \\
        3 & 2 & 4 & 270 \\
        4 & 4 & 3 & 375
    \end{bmatrix}
    & R_2 \gets R_2 - 3R_1, R_3 \gets R_3 - 4R_1 \\
    = & \begin{bmatrix}[ccc|c]
        1 & 1 & 1 & 100 \\
        0 & -1 & 1 & -30 \\
        0 & 0 & -1 & -25
    \end{bmatrix}
    & R_2 \gets (-1)R_2, R_3 \gets (-1)R_3 \\
    = & \begin{bmatrix}[ccc|c]
        1 & 1 & 1 & 100 \\
        0 & 1 & -1 & 30 \\
        0 & 0 & 1 & 25
    \end{bmatrix}
    & \text{Cont.} \longrightarrow
    \end{align*}
\end{frame}

\begin{frame}{Eliminación Gaussiana (Cont.)}{Ecuaciones y Matrices}
    \begin{align*}
        = & \begin{bmatrix}[ccc|c]
            1 & 1 & 1 & 100 \\
            0 & 1 & -1 & 30 \\
            0 & 0 & 1 & 25
        \end{bmatrix}
        & R_2 \gets R_2 + R_3, R_1 \gets R_1 - R_3 \\
        = & \begin{bmatrix}[ccc|c]
            1 & 1 & 0 & 75 \\
            0 & 1 & 0 & 55 \\
            0 & 0 & 1 & 25
        \end{bmatrix}
        & R_1 \gets R_1 - R_2 \\
        = & \begin{bmatrix}[ccc|c]
            1 & 0 & 0 & 20 \\
            0 & 1 & 0 & 55 \\
            0 & 0 & 1 & 25
        \end{bmatrix} \blacksquare
    \end{align*}

    Por tanto, podemos concluir que $coca = 20, burger = 55$ y $papas = 25 \quad \blacksquare$

\end{frame}

\begin{frame}{Matrices Invertibles}{Ecuaciones y Matrices}
    Ahora que ya sabemos lo que es la \textbf{eliminación} y la \textbf{reducción}, podemos contestar la pregunta\dots ¿cómo revierto una multiplicación de matrices? \pause

    \bigskip

    Si $C$ es una matriz de $2 \times  2$, entonces si hacemos $C \times \begin{bmatrix*}
        3 & 0 \\ 0 & 3
    \end{bmatrix*}$ sabemos que obtendremos el triple de $C$. Esto \textit{podría} ser reversible. \pause

    \bigskip

    ¿Qué pasa si tenemos $C \times \begin{bmatrix*}
        0 & 0 \\ 0 & 0
    \end{bmatrix*}$? \pause

    \bigskip

    ¿Es esto reversible?

\end{frame}

\begin{frame}{Matriz Invertible}{Ecuaciones y Matrices}
    Como vimos, hay matrices que no se pueden invertir (por ejemplo, la matriz cuadrada de puros 0s). A este tipo de matrices les llamamos \textit{singulares} o \textit{no invertibles}. \pause

    \bigskip

    Sin embargo, hay otras matrices \textbf{cuadradas} que sí podemos revertir. Si $A$ es una matriz cuadrada, es \alert{invertible} (o \textit{no singular}) siempre que cumpla la siguiente regla: \pause

    \bigskip

    \[AB = I \text{ y } BA = I\]

    \bigskip

    ¿A qué operación te recuerda esto?  

\end{frame}

\begin{frame}{Inversión de matrices}{Ecuaciones y matrices}

    Digamos que $A = \begin{bmatrix*}
        1 & 2 \\ 3 & -5
    \end{bmatrix*}$, y estamos buscando una matriz $B$ \textbf{de las mismas dimensiones} de tal manera que $AB = I_{2 \times 2}$: \pause

    \bigskip

    \[
        \begin{bmatrix*}
            1 & 2 \\
            3 & -5
        \end{bmatrix*}
        \begin{bmatrix*}
            b_{11} & b_{12} \\
            b_{21} & b_{22}
        \end{bmatrix*} =
        \begin{bmatrix*}
            1 & 0 \\ 0 & 1
        \end{bmatrix*}
    \] \pause

    \bigskip

    ¿Cuáles son las ecuaciones que tenemos para cada una de las casillas en $I$?
\end{frame}

\begin{frame}{Inversión de matrices}{Ecuaciones y matrices}
        \begin{align}
            1 \cdot b_{11} - 2 \cdot b_{21} = 1 \\
            3 \cdot b_{11} - 5 \cdot b_{21} = 0 \\
            1 \cdot b_{12} - 2 \cdot b_{22} = 0 \\
            3 \cdot b_{12} - 5 \cdot b_{22} = 1
        \end{align} \pause

        ¿Puedes resolver el sistema? \pause \quad \quad {\scriptsize(Spoiler alert: sí puedes)}

        \bigskip

        Si el sistema tiene solución, entonces las \textit{variables} serán usadas para construir una nueva matriz que es la \alert{inversa} de $A$.
\end{frame}

% Específicamente, la regla 3 puede verse como $R_i = R_i - \frac{a_{ik}}{a_{jk}}R_j$.

% Los robots
% why is it important
% does it exist in math?
% how to represent it
% how to represent it in matlab
% practical cases

% \section*{Referencias}

% \begin{frame}[t]{Referencias}
    % \nocite{bibID01}
    % \nocite{bibID02}

    % \bibliographystyle{IEEE}
    % \bibliography{biblio}
% \end{frame}

\end{document}