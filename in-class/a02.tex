\documentclass[spanish, 10pt]{article}

\usepackage[table, xcdraw]{xcolor}
\usepackage[utf8]{inputenc}
\usepackage[spanish, mexico]{babel}
\usepackage{helvet}
\usepackage{fullpage}
\usepackage{graphicx}
\usepackage{enumitem}
\usepackage{tikz}
\usepackage{ulem}
\usepackage{url}
\usepackage{hyperref}
\usepackage[margin = 3 cm]{geometry}
\usepackage{amsmath}
\usepackage{amsfonts}
\usepackage{gensymb}

\usepackage{matlab-prettifier}
\usepackage{multicol}

\usetikzlibrary{arrows, shapes, trees, calc, decorations.pathreplacing, shapes.misc, positioning, automata}

\setlength\parindent{0pt}

\renewcommand{\familydefault}{\sfdefault}
\newcommand{\responserule}{{\large\rule{14 cm}{0.3mm}}}
\newcommand{\shortresponserule}{{\large\rule{5 cm}{0.3mm}}}
\newcommand{\veryshortresponserule}{{\large\rule{3 cm}{0.3mm}}}
\newcommand{\matlab}[1]{\lstinline[style=Matlab-pyglike]!#1!}


% Specifications for listing package
% \lstset{	
%     basicstyle = \scriptsize\ttfamily,
%     keywordstyle = \color{blue}\ttfamily,
%     stringstyle = \color{red}\ttfamily,
%    	commentstyle = \color{gray}\ttfamily,
%    	tabsize = 3,
%    	breaklines = true,
%    	stepnumber = 1,
%    	showtabs = false,
%    	showstringspaces = false,
%    	frame = none
% }

% Commands for true/false questions
% ----------------------------------------------------------------
\newcommand{\question}[1]{%
	\noindent
	\begin{minipage}[t]{0.15\linewidth}
	\centering		
		\textbf{[\hspace{1 cm}]}
	\end{minipage}%
	\begin{minipage}[t]{0.85\linewidth}
		#1
	\end{minipage}
	\smallskip
}
% ----------------------------------------------------------------

\setlength\parindent{0pt}

\begin{document}

\begin{center}
	{\Large \textbf{Aplicación de Métodos Numéricos al Ambiente Construido (CV1012)}}
	
	\bigskip
	{\large \textbf{Actividad 02 -- Cálculos con Matlab}}
\end{center}

\bigskip
{\large \textbf{Nombre}: \rule{13.7 cm}{0.4mm}}

% \bigskip
% {\large \textbf{Matrícula}: \rule{5 cm}{0.4mm}}

% \bigskip
% {\large \textbf{Name}: \rule{14 cm}{0.4mm}}

\bigskip
{\large \textbf{Matrícula}: \rule{5 cm}{0.4mm}} \hfill {\large \textbf{Fecha}: \today}

\bigskip

\section{Operaciones, cálculos y derivadas}

Genera un script de MATLAB (\textbf{un solo archivo} de extensión \texttt{.m}) que calcule y muestre en pantalla \textbf{todas} las siguientes operaciones.

\subsection{Aritmética}

\begin{enumerate}[label=\alph*)]
	\item $\dfrac{782}{2123} \div \dfrac{919}{1333} =$
	\item $32935872 - 9920175 =$
	\item $\left( \dfrac{2}{7} \right)^{7} =$
	\item $12345 - 7675 \times 345 - 248 =$
	\item $2 \pi =$
	\item $3 \mathrm{e} =$
	\item $\sin 67 \degree =$
	\item $\sqrt{2} =$
\end{enumerate}

\subsection{Derivadas}

Es recomendable que busques la sintaxis y forma de uso de los commandos \matlab{syms}, \matlab{diff}.

\begin{enumerate}[label=\alph*)]
    \item Obtén la derivada de $2x + 3$
    \item Obtén la derivada de $4x^3 + 35x^2 - 10x + 25$
    \item Obtén $f'(x)$ si $f(x) = \sin 3x + x^3$
    \item Obtén la derivada de $\frac{1}{x}$
    \item Obtén $g'(x)$ si $g(x) = e^{-4x} + 70x^2 - 3x + 12$
\end{enumerate}

\subsection{Usando derivadas para cálculos}

Es recomendable que busques primero la sintaxis y forma de uso de los comandos \matlab{subs} y \matlab{vpa}

Sean $f$ y $g$ los incisos \textbf{c} y \textbf{e} de la sección anterior. Obtén los siguientes valores:

\begin{enumerate}
    \item Obtén $f''(x)$ y evalúa en $x=5$
    \item $g'''(x)$ y evalúa en $x=2$
\end{enumerate}

\section{Comandos}

Escribe los símbolos y comandos de MATLAB/Octave que consideres útiles para recordar lo visto en la sesión, y una descripción breve de cada uno de ellos:

\vfill

\textbf{Apegándome al Código de Ética de los Estudiantes del Tecnológico de Monterrey, me comprometo a que mi actuación en esta actividad esté regida por la honestidad académica.}

\end{document}