\documentclass[spanish, 10pt]{article}

\usepackage[table, xcdraw]{xcolor}
\usepackage[utf8]{inputenc}
\usepackage[spanish, mexico]{babel}
\usepackage{helvet}
\usepackage{fullpage}
\usepackage{graphicx}
\usepackage{enumitem}
\usepackage{tikz}
\usepackage{ulem}
\usepackage{url}
\usepackage{hyperref}
\usepackage[margin = 3 cm]{geometry}
\usepackage{amsmath}
\usepackage{amsfonts}

\usepackage{matlab-prettifier}
\usepackage{multicol}

\usetikzlibrary{arrows, shapes, trees, calc, decorations.pathreplacing, shapes.misc, positioning, automata}

\setlength\parindent{0pt}

\renewcommand{\familydefault}{\sfdefault}
\newcommand{\responserule}{{\large\rule{14 cm}{0.3mm}}}
\newcommand{\shortresponserule}{{\large\rule{5 cm}{0.3mm}}}
\newcommand{\veryshortresponserule}{{\large\rule{3 cm}{0.3mm}}}
\newcommand{\matlab}[1]{\lstinline[style=Matlab-pyglike]!#1!}


% Specifications for listing package
% \lstset{	
%     basicstyle = \scriptsize\ttfamily,
%     keywordstyle = \color{blue}\ttfamily,
%     stringstyle = \color{red}\ttfamily,
%    	commentstyle = \color{gray}\ttfamily,
%    	tabsize = 3,
%    	breaklines = true,
%    	stepnumber = 1,
%    	showtabs = false,
%    	showstringspaces = false,
%    	frame = none
% }

% Commands for true/false questions
% ----------------------------------------------------------------
\newcommand{\question}[1]{%
	\noindent
	\begin{minipage}[t]{0.15\linewidth}
	\centering		
		\textbf{[\hspace{1 cm}]}
	\end{minipage}%
	\begin{minipage}[t]{0.85\linewidth}
		#1
	\end{minipage}
	\smallskip
}
% ----------------------------------------------------------------

\setlength\parindent{0pt}

\begin{document}

\begin{center}
	{\Large \textbf{Aplicación de Métodos Numéricos al Ambiente Construido (CV1012)}}
	
	\bigskip
	{\large \textbf{Actividad EX 01 -- Gráficas en MATLAB (+1)}}
\end{center}

\bigskip
{\large \textbf{Nombre}: \rule{13.7 cm}{0.4mm}}

% \bigskip
% {\large \textbf{Matrícula}: \rule{5 cm}{0.4mm}}

% \bigskip
% {\large \textbf{Name}: \rule{14 cm}{0.4mm}}

\bigskip
{\large \textbf{Matrícula}: \rule{5 cm}{0.4mm}} \hfill {\large \textbf{Fecha}: \today}

% {\footnotesize Nota: esta actividad es de mayor dificultad a las que hemos visto anteriormente y probablemente haya dudas.
% Si hay algo que no entiendas, no te quedes sin preguntar.}

\section*{Comandos útiles de la sesión}

{\small \slshape Esta actividad sólo se calificará como entregada o no entregada.
Es para que puedas identificar fácilmente cómo hacer gráficas y cómo manipularlas en MATLAB.}

\bigskip

\matlab{plot(x)} \hfill \\[1.25ex]
\responserule

\bigskip

\matlab{plot(x, y)} \hfill \\[1.25ex]
\responserule

\bigskip

\matlab{plot(x,y,'r--')} \hfill \\[1.25ex]
\responserule

\bigskip

\matlab{plot(x,y,'bd', 'MarkerSize', 5)} \hfill \\[1.25ex]
\responserule

\bigskip

\matlab{hold} \hfill \\[1.25ex]
\responserule

\bigskip

\matlab{legend} \hfill \\[1.25ex]
\responserule

\bigskip

\matlab{legend('Weight', 'Height')} \hfill \\[1.25ex]
\responserule

\bigskip

\matlab{xlabel('Day')} \hfill \\[1.25ex]
\responserule

\bigskip

\matlab{ylabel('Degs')} \hfill \\[1.25ex]
\responserule

\bigskip

\matlab{xlim([-10, 10])} \hfill \\[1.25ex]
\responserule
\bigskip

\matlab{ylim([0, 8])} \hfill \\[1.25ex]
\responserule

% \vfill

% \textbf{Apegándome al Código de Ética de los Estudiantes del Tecnológico de Monterrey, me comprometo a que mi actuación en esta actividad esté regida por la honestidad académica.}

\end{document}