\documentclass[spanish, 10pt]{article}

\usepackage[table, xcdraw]{xcolor}
\usepackage[utf8]{inputenc}
\usepackage[spanish, mexico]{babel}
\usepackage{helvet}
\usepackage{fullpage}
\usepackage{graphicx}
\usepackage{enumitem}
\usepackage{tikz}
\usepackage{ulem}
\usepackage{url}
\usepackage{hyperref}
\usepackage[margin = 3 cm]{geometry}
\usepackage{amsmath}
\usepackage{amsfonts}

\usepackage{matlab-prettifier}
\usepackage{multicol}

\usetikzlibrary{arrows, shapes, trees, calc, decorations.pathreplacing, shapes.misc, positioning, automata}

\setlength\parindent{0pt}

\renewcommand{\familydefault}{\sfdefault}
\newcommand{\responserule}{{\large\rule{14 cm}{0.3mm}}}
\newcommand{\shortresponserule}{{\large\rule{5 cm}{0.3mm}}}
\newcommand{\veryshortresponserule}{{\large\rule{3 cm}{0.3mm}}}
\newcommand{\matlab}[1]{\lstinline[style=Matlab-pyglike]!#1!}


% Specifications for listing package
% \lstset{	
%     basicstyle = \scriptsize\ttfamily,
%     keywordstyle = \color{blue}\ttfamily,
%     stringstyle = \color{red}\ttfamily,
%    	commentstyle = \color{gray}\ttfamily,
%    	tabsize = 3,
%    	breaklines = true,
%    	stepnumber = 1,
%    	showtabs = false,
%    	showstringspaces = false,
%    	frame = none
% }

% Commands for true/false questions
% ----------------------------------------------------------------
\newcommand{\question}[1]{%
	\noindent
	\begin{minipage}[t]{0.15\linewidth}
	\centering		
		\textbf{[\hspace{1 cm}]}
	\end{minipage}%
	\begin{minipage}[t]{0.85\linewidth}
		#1
	\end{minipage}
	\smallskip
}
% ----------------------------------------------------------------

\setlength\parindent{0pt}

\begin{document}

\begin{center}
	{\Large \textbf{Aplicación de Métodos Numéricos al Ambiente Construido (CV1012)}}
	
	\bigskip
	{\large \textbf{Actividad 03 -- Vectores y Matrices}}
\end{center}

\bigskip
{\large \textbf{Nombre}: \rule{13.7 cm}{0.4mm}}

% \bigskip
% {\large \textbf{Matrícula}: \rule{5 cm}{0.4mm}}

% \bigskip
% {\large \textbf{Name}: \rule{14 cm}{0.4mm}}

\bigskip
{\large \textbf{Matrícula}: \rule{5 cm}{0.4mm}} \hfill {\large \textbf{Fecha}: \today}

\bigskip

% {\footnotesize Nota: es probable que esta actividad nos asuste un poco al principio. Es perfectamente normal.
% En efecto, es de mayor dificultad a las que hemos visto anteriormente y probablemente haya dudas.
% Si hay algo que no entiendas, no te quedes sin preguntar.}

\section{Vectores}

Resuelve las operaciones y contesta correctamente. Puedes usar MATLAB/Octave para ayudarte.

\vspace{3ex}

\begin{multicols}{2}
Sean $\mathbf{x} = \begin{bmatrix} 1\\ 3\\ 5\\ 7\\ 9\\ 11\\ 13 \end{bmatrix}$ \quad y \quad $\mathbf{y} = \begin{bmatrix}
    2 \\ 4 \\ 6 \\ 8 \\ 10 \\ 12 \\ 14
\end{bmatrix}$
    \columnbreak
    \begin{enumerate}
        \itemsep2.5ex
        \item $\mathbf{x}_1 =$ \hfill \shortresponserule
        \item $\mathbf{x}_3 =$ \hfill \shortresponserule
        \item $\mathbf{x}_5 =$ \hfill \shortresponserule
        \item $2\mathbf{x}_4 =$ \hfill \shortresponserule
        \item $\mathbf{x}_5^2 =$ \hfill \shortresponserule
        \item $\mathbf{x} + 10 =$ \hfill \shortresponserule
        \item $3 \mathbf{x} =$ \hfill \shortresponserule
        \item $\mathbf{x} + \mathbf{y}$ \hfill \shortresponserule
    \end{enumerate}
\end{multicols}

\section{Matrices}

Antes de comenzar con matrices, hay que hacernos algunas preguntas:

\begin{itemize}
    \item ¿Qué es una variable? \hfill \shortresponserule
    \item ¿Qué es un arreglo? \hfill \shortresponserule
	\item ¿Qué es una matriz? \hfill \shortresponserule
\end{itemize}

\pagebreak

Considera ahora el siguiente problema.

Una patrulla consta de un robot aéreo y uno acuático.
Si queremos hacer una patrulla con un \textit{Apis IV} y un \textit{Myxini II}, ¿Cuántas placas de material necesitamos?

\bigskip

\begin{enumerate}
    \itemsep15ex
    \item Escribe la matriz $P$ de requerimientos de los procesadores.
    \item Escribe la matriz $R$ de requerimientos del \textit{Apis IV} y el \textit{Myxini II}.
    \item Desarrolla la multiplicación a continuación. ¿Da igual si multiplicas $PR$ que $RP$?
    \item ¿Qué operación de las vistas debo aplicar al resultado anterior para saber las placas necesarias para un escuadrón que tiene tres patrullas?
\end{enumerate}

\section{Reflexión}

Escribe los conceptos, tips o símbolos que consideres útiles para recordar lo visto en la sesión. Esta hoja te será de utilidad durante el examen.

\vfill

\textbf{Apegándome al Código de Ética de los Estudiantes del Tecnológico de Monterrey, me comprometo a que mi actuación en esta actividad esté regida por la honestidad académica.}

\end{document}