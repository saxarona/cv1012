\documentclass[spanish, 10pt]{article}

\usepackage[table, xcdraw]{xcolor}
\usepackage[utf8]{inputenc}
\usepackage[spanish, mexico]{babel}
\usepackage{helvet}
\usepackage{fullpage}
\usepackage{graphicx}
\usepackage{enumitem}
\usepackage{tikz}
\usepackage{ulem}
\usepackage{url}
\usepackage{hyperref}
\usepackage[margin = 3 cm]{geometry}
\usepackage{amsmath}
\usepackage{amsfonts}

\usepackage{matlab-prettifier}
\usepackage{multicol}

\usetikzlibrary{arrows, shapes, trees, calc, decorations.pathreplacing, shapes.misc, positioning, automata}

\setlength\parindent{0pt}

\renewcommand{\familydefault}{\sfdefault}
\newcommand{\responserule}{{\large\rule{14 cm}{0.3mm}}}
\newcommand{\shortresponserule}{{\large\rule{5 cm}{0.3mm}}}
\newcommand{\veryshortresponserule}{{\large\rule{3 cm}{0.3mm}}}
\newcommand{\matlab}[1]{\lstinline[style=Matlab-pyglike]!#1!}


% Specifications for listing package
% \lstset{	
%     basicstyle = \scriptsize\ttfamily,
%     keywordstyle = \color{blue}\ttfamily,
%     stringstyle = \color{red}\ttfamily,
%    	commentstyle = \color{gray}\ttfamily,
%    	tabsize = 3,
%    	breaklines = true,
%    	stepnumber = 1,
%    	showtabs = false,
%    	showstringspaces = false,
%    	frame = none
% }

% Commands for true/false questions
% ----------------------------------------------------------------
\newcommand{\question}[1]{%
	\noindent
	\begin{minipage}[t]{0.15\linewidth}
	\centering		
		\textbf{[\hspace{1 cm}]}
	\end{minipage}%
	\begin{minipage}[t]{0.85\linewidth}
		#1
	\end{minipage}
	\smallskip
}
% ----------------------------------------------------------------

\setlength\parindent{0pt}

\begin{document}

\begin{center}
    {\Large \textbf{Aplicación de Métodos Numéricos al Ambiente Construido (CV1012)}}
	
	\bigskip
	{\large \textbf{Actividad 04 -- Estructuras de Datos}}
\end{center}

\bigskip
{\large \textbf{Nombre}: \rule{13.7 cm}{0.4mm}}

% \bigskip
% {\large \textbf{Matrícula}: \rule{5 cm}{0.4mm}}

% \bigskip
% {\large \textbf{Name}: \rule{14 cm}{0.4mm}}

\bigskip
{\large \textbf{Matrícula}: \rule{5 cm}{0.4mm}} \hfill {\large \textbf{Fecha}: \today}

\bigskip

% {\footnotesize Nota: es probable que esta actividad nos asuste un poco al principio. Es perfectamente normal.
% En efecto, es de mayor dificultad a las que hemos visto anteriormente y probablemente haya dudas.
% Si hay algo que no entiendas, no te quedes sin preguntar.}

\section{Arreglos}

Resuelve las operaciones y contesta correctamente. Puedes usar MATLAB/Octave para ayudarte.

\begin{enumerate}
    \itemsep2.5ex
    \item Sean $\mathbf{x} = \langle 1, 3, 5, 7, 9, 11, 13 \rangle$ y $f(x) = x^3$
        \bigskip
    \begin{enumerate}
        \item $f(\mathbf{x}_1) =$ \quad \; \shortresponserule
        \item $f(\mathbf{x}_2) =$ \quad \; \shortresponserule
        \item $f(\mathbf{x}_3) =$ \quad \; \shortresponserule
        \item $f(\mathbf{x}_4) =$ \quad \; \shortresponserule
        \item $f(\mathbf{x}_5) =$ \quad \; \shortresponserule
        \item $f(\mathbf{x}_6) =$ \quad \; \shortresponserule
        \item $f(\mathbf{x}_7) =$ \quad \; \shortresponserule
    \end{enumerate}
    \item $\mathbf{x}_2^2 =$ \hfill \shortresponserule
    \item $\mathbf{x}_7^4 =$ \hfill \shortresponserule
    \item $\mathbf{x}_i+2 =$  \hfill \shortresponserule
    \item $\mathbf{x}_i^2 =$  \hfill \shortresponserule
\end{enumerate}

\section{Matrices}

Considera ahora la siguiente matriz:

$$A = \begin{bmatrix}
    1 & 2 & 3 & 4\\
    5 & 6 & 7 & 8\\
    9 & 10 & 11 & 12\\
    13 & 14 & 15& 16
\end{bmatrix}$$

\begin{itemize}
    \item ¿Cuántas \textbf{filas} tiene $A$? \hfill \shortresponserule
    \item ¿Cuántas \textbf{columnas} tiene $A$? \hfill \shortresponserule
    \item ¿Cuál es el elemento de la fila 1 y de la columna 3? \hfill \shortresponserule
    \item ¿Cuál es el elemento $A_{4,2}$? \hfill \shortresponserule
\end{itemize}

\begin{itemize}
    \item $A_{1} =$ \hfill \shortresponserule
    \item $A_{3,1} =$ \hfill \shortresponserule
    \item $A + 10 =$
    \vspace{20ex}
    \item $A^2 =$
    \vspace{20ex}
\end{itemize}

\section{Comandos}

Escribe los símbolos y comandos de MATLAB/Octave que consideres útiles para recordar lo visto en la sesión, y una descripción breve de cada uno de ellos:

\vfill

\textbf{Apegándome al Código de Ética de los Estudiantes del Tecnológico de Monterrey, me comprometo a que mi actuación en esta actividad esté regida por la honestidad académica.}

\end{document}